\chapter{Quaternion}
\label{chap: Quaternion}
In software voor 3D-applicaties wordt niet de rotatie gebruikt zoals in hoofdstuk \ref{chap: matrix, rotatie en projectie} beschreven. Daar wordt de methode van Euler gebruikt, maar die levert problemen op als je over 3 assen roteert. Dat probleem wordt de Gimbal Lock genoemd. Zoek even op you tube naar 'Gimbal Lock'. 
Een oplossing voor deze problemen bieden de zogenaamde quaternionen. En daarvoor moeten we eerst even iets over complexe getallen uitleggen.
\section{Complexe getallen}
Complexe getallen zijn getallen waarvan je vroeger misschien geleerd hebt dat ze niet bestaan. Immers als je $ \sqrt{-1} $ opschreef zei je leraar waarschijnlijk: 'Dat kan niet'. Toch zijn complexe getallen (en quaternionen) zeer handige 'rekenhulpjes' in 3d simulaties en ook in electrotechniek. En ze zijn belangrijk in fractals, waarvan je vast de mooie Mandelbrot figuren kent. \\ \\
\mydef[Het imaginaire getal \textit{i} is een getal zó dat:]
{imaginair }{ $  i^2 = -1 $ }\\
Dat is hetzelfde als $ i =\sqrt{-1} $ \\ \\
\mydef[We definieren een complex getal z als:]
{complex getal}{   $  z = a + bi $ \quad met a en b reële getallen}\\
\mybv[complex getal]{$ z = 2+3i  $ is een complex getal en ook $ 1-i. $ \\Let op met rekenen: $ (1-i)^2 = (1-i).(1-i) = 1 - i - i -1 = -2i $ }
We zeggen ook wel dat z bestaat uit een reëel deel \textit{a} en een imaginair deel \textit{bi} .  Je kunt met complexe getallen net zo rekenen als met reële getallen als je maar rekening houdt met $  i^2 = -1.$
\section{Quaternion}
Quaternionen zijn een uitbreiding van complexe getallen. In plaats van één imaginair deel heb je er drie:\\ \\
\mydef[Een quaternion is:]
{quaternion }{ $ q = a +bi + cj + dk $ 
	\qquad a, b, c, d \textbf{\textit{reëel}}
	\qquad  i, j, k  \textbf{\textit{imaginair}}\\ \\ i, j, k voldoen aan de volgende regels:}
$ i^2 = -1 \qquad  \ j^2 = -1 \qquad k^2 = -1   \\
i.j = k \qquad \ \ j.i = -k \\
i.k = -j \qquad k.i = j \\
j.k = i \qquad \ \ k.j = -i \\ $
Met a, b, c en d kun je gewoon rekenen, omdat het reële getallen zijn.\\ \\
\mybv[quaternion]{$ q_1 = 2 - i  + 2j - 3k $ is een quaternion net als $ q_2 = - j + k $ of  $ q_3 = 2i - j + \frac{1}{3}k. $}
\subsection{rekenschema}
Om een beetje vlot te kunnen rekenen met quaternionen heb je een schema nodig:
\begin{center}
	\begin{tabular}{ | l || c | c |c |c |}
		\hline
		& 1 & i & j & k \\ \hline \hline
		1 & 1 & i & j & k \\ \hline
		i & i & -1 & k & -j\\ \hline
		j & j & -k & -1 & i\\ \hline
		k & k & j & -i & -1\\ 
		\hline 
	\end{tabular}
\end{center}

Je zoekt \textit{eerst} in de $ 1^e $ kolom welke van i, j of k je nodig hebt en \textit{daarna} het volgende  imaginaire getal in de $ 1^e $ rij.  Op het kruispunt staat het resultaat van de vermenigvuldiging (de volgorde is belangrijk omdat bv $  i.k = -j $ maar $ k.i = j  $ !) \\ \\
\mybv[vermenigvuldiging quaternionen]{als $ q_1 = 6 + 3i - 5j + 2k $ en $ q_2 = 2 + i + 4j - 2k $ wat is dan $ q_1.q_2 $? }
Daarvoor zijn er 2 manieren:\\

\textbf{1. invullen in tabel} Dit is de handigste manier:  Schrijf de eerste quaternion in de eerste kolom en de tweede quaternion in de eerste rij. Vul op de kruispunten de 
vermenigvuldigingen in met de  regels uit het schema. Bijvoorbeeld op het kruispunt van de $ 3^e $ rij en de $   5^e $ kolom: $  3i\times -2k = -6ik   $ en omdat $ ik = -j $ is de uitkomst $ --6j = +6j $. Tot slot verzamel je  alle i, j en k.
\begin{center}
	\begin{tabular}{ | l || c | c |c |c |}
		\hline
		$ q_1.q_2 $& 2 & i & 4j & -2k \\ \hline \hline
		6 & 12 & 6i & 24j & -12k  \\ \hline
		3i & 6i & -3 & 12k & 6j\\ \hline
		-5j & 10j &  5k & 20 & 10i\\ \hline
		2k & 4k & 2j & -8i & 4\\ 
		\hline 
	\end{tabular}
\end{center}
en dus:
\begin{align*}
q_1.q_2 & = 12 - 3 + 20 + 4 \\
& \ \ \ +(6 + 6 + 10 - 8)i \\
& \ \ \ +(24 + 6 +10 +2)j \\
& \ \ \ +(-12 + 12 + 5 +4 )k \\
& = 33 +14i +42j +9k \\
\end{align*}
\textbf{2. Uitschrijven}: Je schrijft alle vermenigvuldigingen op en kijkt per vermenigvuldiging in de tabel wat er uit komt:
\begin{align*}
q_1.q_2 & = (6 + 3i - 5j + 2k) . (2 + i + 4j - 2k) \\
& =   6.2 + 6i + 6.4j + 6.-2k \qquad  \qquad \qquad  'gewoon' \  vermenigvuldigen\\
& \ \ \ + 3i.2 + 3i.i + 3i.4j + 3i.-2k \qquad \qquad  i\  in \ 1^e \  kolom  \ opzoeken\\
& \ \ \ + -5j.2 + -5j.i + -5j.4j + -5j.-2k \qquad j \  in \ 1^e \ kolom \ opzoeken\\
& \ \ \ + 2k.2 + 2k.i + 2k.4j + 2k.-2k \qquad \qquad k \  in \ 1^e \ kolom \ opzoeken\\
& = 18 + 6i + 24j - 12k \\
& \ \ \ + 6i - 3 + 12k -6.-j \\
& \ \ \ -10j - 5.-k + 20 +10i \\
& \ \ \  + 4k + 2j + 8.-i + 4 \qquad \qquad \qquad   nu \ alle \ i, \ j \ en \ k\ bij \ elkaar \  zoeken \\
& = 18 - 3 + 20 + 4 + (6+6 + 10 -8)i + (24 + 1 - 10 + 2)j + (-12 + 12 + 5 + 4)k \\
&  = 39 + 14i +17j +9k
\end{align*}
\mydef[ ]
{geconjungeerd}
{De geconjungerde van $ q = a +bi + cj + dk $  is: 
	\quad	$ q^* = a - bi - cj - dk $ }\\ \\
\mybv[geconjungeerde]{Stel $ q = 1 - 2i + 3j + k $  dan is $ q^* = 1 + 2i - 3j - k $ \\
	Als $ q =   i - 3j  $  dan is $ q^* = -i + 3j  $}
\section{Roteren met quaternionen}
Het doel van quaternionen is dat we kunnen roteren. Daarvoor moeten we ook punten van \RD omzetten naar quaternionen.\\ \\
\mydef
{puntquaternion}
{Als P = (x, y, z)  in \RD dan is\\
	het puntquaternion dat er bij hoort: \quad	$ p = xi+yj+zk $  
}\\ 
\mybv[puntquaternion]
{
	Stel $ P = (1, 3, -2) $  \ \  dan is  \qquad $ p =0 + 1.i+3j-2k  \ \ = i+3j-2k $ \\ het puntquaternion dat er bij hoort
}\\
Voor het berekenen van een rotatie in \RD geldt de volgende formule: 
\mydef
{quaternionrotatie}
{Als p een punt en $  q_r $ een rotatie is in \RD dan is:\\
	het geroteerde punt \quad	$ p' = q_r.p.q^*_r $  \\ 
}
\mybv[quaternionrotatie]
{Stel $ P = (3,0,1) $ en $ q_r = 2j-k $ wat is dan  $ q_r.p.q^*_r $ ?\\ \\
	$ p = 3i+k  \ \ \ \ \quad = 0 + 3i + 0.j + k $ \\
	$ q^*_r = -2j+k  \quad = 0 + 0.i -2j + k $ \\
	dan is de tabel voor $  p.q^*_r: $ 
	\quad (p in de $  1^e $ kolom, $  q^*_r: $ in de $ 1^e $ rij)
	\begin{center}
		\begin{tabular}{ | l || c | c |c |c |}
			\hline
			$ p.q^*_r $  & 0 & 0   & -2j & k \\ \hline \hline
			0 & 0 & 0   & 0    & 0  \\ \hline
			3i & 0 & 0 & -6k & -3j\\ \hline
			0 & 0 &  0 & 0     & 0\\ \hline
			k & 0 & 0   & 2i   & -1 \\
			\hline 
		\end{tabular}
	\end{center}
	en dus $ p.q^*_r  = -1+2i-3j-6k. $ } 
Daarna met de tabel  $  q_r.(p.q^*_r) $ uitrekenen  
\quad ($  q_r  $ in de $  1^e $ kolom, $  p.q^*_r: $ in de $ 1^e $ rij)\\
\begin{center}
	\begin{tabular}{ | l || c | c |c |c |}
		\hline
		$ q_r.(p.q^*_r) $ & -1 & 2i   & -3j & -6k \\ \hline \hline
		0    & 0 & 0   &   0        & 0  \\ \hline
		0    & 0 & 0   &   0        & 0  \\ \hline
		2j   & -2j &  -4k &   6     & -12i\\ \hline
		-k    & k  & -2j   & -3i   & -6\\ 
		\hline 
	\end{tabular}
\end{center}
en dus is $ p' = q_r.(p.q^*_r) =  -15i -4j -3k. $ Dat betekent dat het punt $ P = (3,0,1) $  onder het quaternion  $ q_r = 2j-k $ afgebeeld wordt op het punt $ P' = (-15, -4, -3). $ Dat is een vreemde rotatie (want de afstand tot de oorsprong wordt ineens groter) . Dat komt omdat we voor $ q_r $ wat gemakkelijke waardes hebben genomen en dat is geen echt rotatiequaternion.\\
Daarom hebben we de volgende definitie nodig:\\ \\
\mydef
{rotatie quaternion}
{Gegeven de eenheidsvector $\hat{v} $  en de  hoek $\alpha $ dan is:\\
	het rotatiequaternion \quad	
	$ q_r = cos\frac{\alpha}{2} + sin\frac{\alpha}{2} .\hat{v}_1.i
	+ sin\frac{\alpha}{2} .\hat{v}_2.j + sin\frac{\alpha}{2} .\hat{v}_3.k $.	    
}\\
\mybv[rotatiequaternion]
{Stel $ \vec{v} = \vecdrie{-3}{0}{4}  $  dan is  
	$\hat{v} \ = \ \frac{1}{5} \ \vecdrie{-3}{0}{4}   $  \quad (omdat $ |\vec{v}| = \sqrt{3^2+ 4^2} = 5  $ )\\
	Stel verder dat we over $\alpha$ = 180\degree willen roteren, dan is $\frac{180\degree}{2} = 90\degree $ \\
	We weten dat $  \cos 90\degree = 0 $ en  $ \sin 90\degree = 1 $ 
	\ \ En dus is 
	\begin{align*}
	q_r &= \cos\frac{180}{2} \ + \ \sin\frac{180}{2} .-\frac{3}{5}.i
	\ + \ \sin\frac{180}{2} .0.j \ + \ \sin\frac{180}{2} .\frac{4}{5}.k   \\
	q_r  &=  \cos 90 - \frac{3}{5}.\sin 90 .i  \ + \ \frac{4}{5}. \sin 90 .k  \\
	q_r  &=  - \frac{3}{5}i  + \frac{4}{5}k 
	\end{align*}
}\\
\mybv[rotatie met quaternionen]
{We nemen dezelfde vector en hoek als hierboven. \\Stel verder dat we $ P=(-1,-1,0)  $  willen roteren. \\Dan is $ p = - i - j $ het puntquaternion.
	\\We hadden  $ q_r  =   - \frac{3}{5}i  + \frac{4}{5}k $  
	\\en dus is  $ q^*_r  =    \frac{3}{5}i  - \frac{4}{5}k. $ 
	\\Eerst moeten we $   p.q^*_r $  uitrekenen:}
\begin{center}
	\begin{tabular}{ | l || c | c |c |c |}
		\hline
		$ p.q^*_r $  & 0 & $  \frac{3}{5}i  $  & 0 & $ - \frac{4}{5}k $  \\ \hline \hline
		0                 & 0 & 0                          & 0    & 0  \\ \hline
		-i                & 0 &  $  \frac{3}{5}  $  & 0   & $ - \frac{4}{5}j $\\ \hline
		-j                & 0 &  $  \frac{3}{5}k $  & 0     & $  \frac{4}{5}i $\\ \hline
		0                 & 0 & 0                          & 0   & 0 \\
		\hline 
	\end{tabular}
\end{center}
En dus is $ p.q^*_r = \frac{3}{5}  + \frac{4}{5}i - \frac{4}{5}j  + \frac{3}{5}k. $
\\Anders geschreven:  $ p.q^*_r = \frac{1}{5}  (3 + 4i - 4j  + 3k). $ 
\\Dit vullen we  in in de $ 1^e $ rij en $ q_r $ in de $ 1^e  $ kolom om $ q_r.(p.q^*_r) $ te berekenen:\\
\begin{center}
	\begin{tabular}{ | l || c | c |c |c |l}
		\hline
		$ q_r.(p.q^*_r) $  & 3 & 4i   & -4j & 3k &  $ \times  \frac{1}{5} $\\ \hline \hline
		0                         & 0    & 0    & 0       & 0 & \\ \hline
		-3i                      & -9i   & 12  & 12k    & 9j & \\ \hline
		0                         & 0    &  0   & 0       & 0&\\ \hline
		4k                       & 12k & 16j  & 16i   & -12 & \\ 
		\hline 
		$ \times  \frac{1}{5} $
	\end{tabular}
\end{center}
Om de berekening overzichtelijk te houden zijn de breuken 'buiten haakjes gehaald'. Zowel voor $ q_r $ als voor $ p.q^*_r $ is dat $  \frac{1}{5}. $ 
\\Dat betekent dat we alles met $  \frac{1}{5} \times  \frac{1}{5}  =  \frac{1}{25} $ moeten vermenigvuldigen: 
\begin{align*}
p' &=  q_r.(p.q^*_r)  \\
&=  \frac{1}{25}(-9i + 12 +12k + 9j + 12k + 16j + 16i -  12)  \\
& = \frac{1}{25} (7i + 25j +  24k  )
\end{align*}
En dat betekent dat het punt $ P=(-1,-1,0)  $ waar we mee begonnen  geroteerd wordt naar $ P'=  (\frac{7}{25}, \frac{25}{25}, \frac{24}{25}) =(0.28,1,0.96)  $, zie  figuur \ref{fig:quatrotatie}.

\figuur[0.6]{quatrotatie}{De rotatie om $ \vec{v} $ over 180\degree. $ P(-1,-1,0) $ rood, wordt geroteerd naar $ P'=(0.28,1,0.96)  $, blauw }
\subsubsection{Opgaven}
\begin{enumerate}
	\item Gegeven $ P (-2, 2, 9) $ en de vector $\vec{v} = \vecdrie{-4}{0}{3} $. 
	We roteren over $ 84\degree. $ Geef het puntquaternion \textit{p} en het geconjungeerde rotatiequaternion  $  q_r^* $.
	\item Gegeven $ P (23, -15, 7) $ en de vector $\vec{v} = \vecdrie{6}{-6}{7} $. 
	We roteren over $ 42\degree. $ Geef het puntquaternion \textit{p} en het geconjungeerde rotatiequaternion  $  q_r^* $.
	
	\item Gegeven $ a = 2i-4j+5k$ en   $ b = 7+2i+j $. 
	Bereken het product \ $  a.b $.
	\item Gegeven $ c = -16 -2i+3j+2k $ en   $ d = -1+4i+2k $. 
	Bereken het product \ $  c.d $.
	
	\item Gegeven $ P (-2, 0, 5) $ en de vector $\vec{v} = \vecdrie{-3}{0}{0} $. 
	We roteren over $ 60\degree. $ Gebruik quaternionen om het beeld van \textit{P }onder deze rotatie uit te rekenen.
	
\end{enumerate}

\subsubsection{extra opgaven}

\begin{enumerate}
	\item Gegeven $ P (2,2,0) $ en de vector $\vec{v} = \vecdrie{4}{-2}{4} $. 
	We roteren over $ 42\degree. $ Geef het puntquaternion \textit{p} en het geconjungeerde rotatiequaternion  $  q_r^* $. 
	
	\item Gegeven $ P (6,-2,3) $ en de vector $\vec{v} = \vecdrie{-12}{12}{6} $. 
	We roteren over $ 84\degree. $ Geef het puntquaternion \textit{p} en het geconjungeerde rotatiequaternion  $  q_r^* $. 
	
	\item Gegeven $ e = 2i-4j+5k$ en   $ f = 7+2i+j $. 
	Bereken het product \ $  e.f $.
	
	\item Gegeven $ g = -16 -2i+3j+2k $ en   $ h = -1+4i+2k $. 
	Bereken het product \ $  g.h $.
	
	\item  Gegeven $ P (-2, 0, 0) $ en de vector $\vec{v} = \vecdrie{0}{7}{0} $. 
	We roteren over $ 60\degree. $ Gebruik quaternionen om het beeld van \textit{P }onder deze rotatie uit te rekenen.   
	
	\item Schrijf een programma dat quaternionen kan vermenigvuldigen.   
	
	\item Schrijf een programma dat quaternionrotatie kan uitvoeren (dus gegeven een willekeurige vector $\vec{v} $ , hoek $ \alpha $ en punt P, kan uitrekenen waar P' onder de rotatie uitkomt).      
	
\end{enumerate}


