\chapter{Spiegeling, translatie en samenstelling}
\label{chap: Spiegeling, translatie en samenstelling}
In dit hoofdstuk gaan we nog een lineaire afbeelding behandelen, namelijk spiegeling. Daarnaast wordt ook translatie behandeld en komen we erachter dat deze niet lineair is. Uiteindelijk gaan we verschillende afbeeldingen combineren (samenstellen).

\section{Spiegeling}
Een spiegeling in \RT wordt uitgevoerd door vanuit een punt een loodlijn te trekken naar de lijn waarin je spiegelt, en vervolgens die loodlijn even lang door te trekken naar de andere kant van de spiegellijn om het beeld van het originele punt te vinden. Zie figuur \ref{fig:spiegeling2D_2}
\figuur[0.7]{spiegeling2D_2}{De spiegeling in de lijn $ y=4x$. Het rode punt is het spiegelbeeld van (1,0), het blauwe het spiegelbeeld van (0,1). De grijze punten zijn de projectiepunten.}

\subsection{De matrix van een spiegeling in \RT}
\mybv[spiegeling \RT] {Als voorbeeld nemen we de spiegeling in de lijn $  y = 4x $ (figuur \ref{fig:spiegeling2D_2}).}  We kunnen bij een spiegeling gebruik maken van wat we bij een projectie uitgerekend hebben (zie 
\textit{de matrix van een projectie in \RT} blz \pageref{projectie2D}). $ P_{10}  $ is het punt waarop (1,0) \textit{geprojecteerd} wordt, en $ S_{10}  $ het punt  waarnaar toe (1,0) \textit{gespiegeld} wordt. Verder stellen we dat 
$\vec{p} = \overrightarrow{OP_{10} }. $ ( $\overrightarrow{OP_{10}} $ is de vector van de oorsprong naar punt $P_{10}$) \\ 
Met behulp van figuur \ref{fig:spiegeling2Dvector} kun je zien dat 
$  \red{ \overrightarrow{OS_{10} } }
=  \blu{ \vec{p}  + - \hat{b}_x+  \vec{p} }
= 2 \vec{p}  - \hat{b}_x. $ \\ \\
Het zal je  niet verbazen dat de volgende regel geldt:\\
Als $\vec{p} $ het beeld van een basisvector $ \hat{b}  $  bij een projectie is dan is $ 2 \vec{p}  - \hat{b} $ het beeld van de bijbehorende spiegeling. Nog korter: \\ \\
\myeig[spiegeling]
{als \quad $ \hat{b}  \  \xrightarrow{P}   \  \vec{p}  \quad  $
	dan \qquad   $ \hat{b}  \  \xrightarrow{S}   \  2\vec{p} \ - \  \hat{b}  $ 
	\qquad \qquad voor elke basisvector $  \hat{b}  $} 

\figuur[1]{spiegeling2Dvector} { Het blauwe pad volgen is hetzelfde als  in één keer de rode vector.  
	En dus geldt: als $  \vec{p} $ het beeld van een projectie is, dan is het spiegelbeeld $ =  2\vec{p} - \hat{b}.  $ \  (bij dit voorbeeld  is $ \hat{b} = \hat{b}_x , $ de basisvector langs de x-as). }  

Van de projectie in \RT weten we dat 
$ \vec{p} = \vectwee{ \nicefrac{1}{17} } { \nicefrac{4}{17} } \  $ \\ \\ \\dus is 
$ 2\cdot \vec{p} -  \hat{b}_x
\  =  \   2\cdot \vectwee{ \nicefrac{1}{17} } { \nicefrac{4}{17} } \ - \ \vectwee{1}{0} 
\ = \    \vectwee{ \nicefrac{2}{17} -1 } {\nicefrac{8}{17}} 
\ = \    \vectwee{ -\nicefrac{15}{17} } {\nicefrac{8}{17}} 
$ \\ \\ \\
en dat betekent dat   $ \vectwee{1}{0}  \  \xrightarrow{S}   
\  \vectwee[red]{ -\nicefrac{15}{17} } {\nicefrac{8}{17}} . $ \\ \\
We passen deze eigenschap ook toe om te vinden wat het beeld van  $  \hat{b}_y\ = \ \vectwee{0}{1} $ onder onze spiegeling is:\\ \\
Er gold:  $ \vectwee{0}{1}   \  \xrightarrow{P}  \ 
\vectwee{\nicefrac{4} {17} } {\nicefrac {16} {17}} $
\qquad  dan is $  \  2\vec{p} \ - \  \hat{b}_y \ 
= 2 \cdot \vectwee{ \nicefrac {4} {17} } { \nicefrac {16} {17}} \ - \ \vectwee{0}{1}  \ 
= \  \vectwee{\nicefrac{8}{17} } {\nicefrac{15}{17}}  $. \\

\quad dus  $ \vectwee{0}{1}   \  \xrightarrow{S}  \
\vectwee[blue]{\nicefrac{8}{17} } {\nicefrac{15}{17} } $ 
\\ \\
En dus is de matrix van 
$ S = \mattwee{ 
	\red{ - \nicefrac{15}{17}} & { \color{blue}\nicefrac{8}{17}} }
{ \red{ \nicefrac{8}{17} } & { \color{blue} \nicefrac{15}{17} } } 
 = \ \frac{1}{17} \cdot   \mattwee{-15&8}{8&15} $\\ \\
Misschien valt je op dat de getallen in de matrix niet toevallig lijken (linksonder en rechtsboven zijn hetzelfde en de andere twee elkaars negatief). Dat is ook niet toevallig. Als je dat handig vindt mag je de uitkomst van een spiegelberekening controleren met de volgende formule:\\

\myeig[spiegeling]
{Als \textit{S} een spiegeling is in de lijn $ y=a\cdot x $, dan is de matrix van de spiegeling\\
	$ S = \frac{1}{a^2+1}  \mattwee{ 
		\red{ 1-a^2} & { \color{blue} 2a} }
	{ \red{ 2a } & { \color{blue} a^2-1} } $ }

\subsection{De matrix van een spiegeling in \RD}
Net zoals we bij een spiegeling in \RT gebruik maken van de projectie (op dezelfde lijn waarin we spiegelen), kunnen we voor een spiegeling in \RD gebruik maken van de projectie op het vlak waarin we willen spiegelen. Want er geldt dezelfde regel als in \RT:\\ \\

\myeig[spiegeling]
{als $ \hat{b}  \  \xrightarrow{P}   \  \vec{p}  \quad  $
	dan \qquad   $ \hat{b}  \  \xrightarrow{S}   \  2\vec{p} \ - \  \hat{b}  $  
	\qquad \qquad voor elke basisvector $  \hat{b}  $ } \\ 

\mybv[spiegeling  \RD] {Als voorbeeld nemen we de spiegeling in het vlak $ W:  2x-3y+z = 0. $ }
Van de \textit{projectie op $W$}  (zie:  'matrix van een projectie in \RD' op blz.\pageref{projectieR3}) weten we dat:\\ \\
$ \ \vecdrie{1}{0}{0}   \  \xrightarrow{P}  \  \frac{1}{7}\vecdrie {5 } {3}{-1} $
\quad   $ \ \vecdrie{0}{1}{0}   \  \xrightarrow{P}  \ \frac{1}{14}  \vecdrie{ 6 }{5 }{-3 }\ $ 
en \quad  
$  \ \vecdrie{0}{0}{1}   \  \xrightarrow{P}  \  \frac{1}{14}  \vecdrie{2 }{-3 }{15}\ $\\ \\ \\
dat betekent dat bij de spiegeling geldt:\\
voor $ \hat{b}_x $ is \  
$\vec{p} =  \frac{1}{7} \ \vecdrie {5 } {3}{-1} $\quad  en \quad 
$ 2\vec{p}  -  \hat{b}_x  = 
\ 2 \cdot  \frac{1}{7} \vecdrie {5 } {3}{-1} \  - \  \vecdrie{1}{0}{0}  \ =
\   \frac{1}{7} \ \vecdrie {3 } {6}{-2} $ \\en dus 
$ \ \vecdrie{1}{0}{0}   \  \xrightarrow{S}  \  { \color{red}\frac{1}{7}} \  \vecdrie[red] {3 } {6}{-2} $\\
Voor $ \hat{b}_y  $ geldt:
$ 2\vec{p}  -  \hat{b}_y  =  
\ \frac{2}{14} \ \vecdrie {6 } {5}{-3} \  - \  \vecdrie{0}{1}{0}  \ = 
\   \frac{1}{7} \ \vecdrie {6 } {-2}{-3} $  \\ en dus  \quad 
$ \ \vecdrie{0}{1}{0}   \  \xrightarrow{S}  \  { \color{blue}\frac{1}{7}} \  \vecdrie[blue] {6 } {-2}{-3} $\\
en voor $ \hat{b}_z  $ geldt:
$ 2\vec{p}  -  \hat{b}_z  =  
\ \frac{2}{14} \ \vecdrie {2 } {-3}{15} \  - \  \vecdrie{0}{0}{1}  \ = 
\   \frac{1}{7} \ \vecdrie {2 } {-3}{8} $   \\ en dus 
$ \ \vecdrie{0}{0}{1}   \  \rightarrow{S}  \  { \color{green}\frac{1}{7}} \  \vecdrie[green] {2} {-3}{8}. $\\
Dan is de matrix van 
$ S = 
\frac{1}{7} \matdrie{  \red{3} &  \blu{6} &   \gre{2} }
{   \red{6}   &   \blu{-2} &  \gre{-3} }
{  \red{-2} &   \blu{-3}  &   \gre{8}} $

\section{Translatie}		
Transleren, anders gezegd verschuiven, is een afwijkende afbeelding. Zoals we zullen zien is transleren, in tegenstelling tot roteren, spiegelen en projecteren \textit{niet} lineair. Dat is belangrijk want daarom kunnen we er niet zomaar een matrix van maken. En alleen als we een matrix hebben kunnen we (beter gezegd software) er  goed mee rekenen. 
\subsection{Translatie niet lineair}
\mybv[translatie \RT] {Als voorbeeld nemen we de translatie over de vector  $ \vec{t}=  \vectwee{2}{1}. $ }
\figuur[0.8]{translatie2D2}{De translatie over de vector  $ \vec{t} $. }
Je kunt de werking van deze translatie als volgt opschrijven:\\ neem een willekeurige vector $ \vectwee{x}{y} $ dan is het beeld daarvan $ T \vectwee{x}{y} = \vectwee{x+2}{y+1} $.\\ \\ \\
Je kunt nu op verschillende manieren zien dat een matrix "niet werkt":\\ \\
\textbf{ Ten eerste:} $ T \vectwee{1}{0} = \vectwee{1+2}{0+1} = \vectwee{3}{1} $ en dat \textit{\textbf{zou}} betekenen dat 
$ \vectwee{1}{0}  \  \xrightarrow{T}   \   \vectwee[red]{ 3 } {1}. $ \\ Op dezelfde manier \textit{\textbf{zou}} je zien dat  
$ \vectwee{0}{1}  \  \xrightarrow{T}   \   \vectwee[blue]{ 2 } {2}. $ \\ Dat \textbf{\textit{zou}} betekenen dat 
$ T = \mattwee{ 3 & 2 }{ 1 & 2 } $ En \textbf{\textit{als}} dat zo zou zijn \textbf{\textit{zou}} 
$  T \vectwee{0}{0} \ = \ 
\ \vectwee{ 3\cdot 0 + 2\cdot 0 }{ 1\cdot 0 + 2\cdot 0 } \ 
= \ \vectwee{0}{0} $ Dat \textbf{\textit{zou}} betekenen dat het de oorsprong (0,0) niet verschoven zou worden! Overigens heet zo'n redenering als deze  $"$een bewijs uit het ongerijmde".\\ \\ \\
\textbf{Ten tweede:} als $T$ lineair zou zijn zou moeten gelden\\ 
$ T(\vec{a} \ + \ \vec{b} )  
=  T(\vec{a}) \ + \ T(\vec{b}) $ \\ dus bijvoorbeeld voor $ \vec{a} = \vectwee{1}{0} $ en   
$ \vec{b} \ = \ \vectwee{0}{1} $  is \\ 
$ T(\vec{a} \ + \ \vec{b} ) \ 
= \ T \ \vectwee{1+0}{0+1} \ 
= \  \vectwee{1+2}{1+1} \ 
= \ \vectwee{3}{2}  $ 
\quad maar \\
$ T(\vec{a}) \ + \ T(\vec{b}) 
= \  \vectwee{1+2}{0+1} \  + \  \vectwee{0+2}{1+1} \ 
= \  \vectwee{3}{1} \ + \  \vectwee{2}{2} \ 
= \  \vectwee{5}{3} \ 
$\\ \\
Kortom 
\myeig[lineariteit]{\textbf{\textit{een translatie is niet lineair}}} 

\subsection{Affiene  matrix van translatie in \RT}
Maar er bestaat een truc om  toch met een translatie te kunnen rekenen. We voegen aan de translatie een dimensie toe. Dat wil zeggen voor een twee-dimensionale translatie maken we een drie-dimensionale matrix en voor een drie-dimensionale translatie maken we een  vier-dimensionale matrix:\\ \\ 
\mydef [Gegeven  een translatie vector]
{affiene matrix} { $\vec{t} = \vectweesnel{t} $  \quad  dan is 
	$ T_a = \matdrie{ 1 & 0 & t_1 }
	{ 0 & 1 & t_2 }
	{0 & 0 & 1 } $ \quad de affiene matrix  } \\
\mybv[affiene matrix]{voor de translatie over $ \vec{t}=  \vectwee{2}{-3} $ is de affiene matrix 
	$ T_a = \matdrie{ 1 & 0 & 2 }
	{ 0 & 1 & -3 }
	{0 & 0 & 1 } $  }
\subsection{Rekenen met een affiene  matrix in \RT}
Om te kunnen rekenen met een affiene matrix moeten we aan alles een dimensie toevoegen: aan punten, vectoren en andere matrices. Dat doen we als volgt: 
\mydef
{affiene vector}{Als  $\vec{v} = \ \vectweesnel{v} $ dan  is 
	$\vec{v}_a = \  \vecdrie{v_1}{v_2}{1}  $ \quad de \textit{affiene} vector } \\
\mybv[affiene vector]{Bijv. als $ \vec{t}=  \vectwee{-4}{7} $  dan is  de affiene vector 
	$ t_a = \matdrie{ -4 }	{ 7 }	{1 }$ } \\
Nu kunnen we  controleren of de affiene matrix voor $T$ klopt. We nemen weer de basisvector langs de x-as $ \hat{b}_x $ en voegen ook daar een dimensie aan toe! \\
$ \hat{b}_x \ = \ \vectwee{1}{0} $ en dus is  
$ \hat{b}_{xa} \ = \  \vecdrie{1}{0}{1}. $ \\
$T_a$ is in ons geval bij $ \vec{t}=  \vectwee{2}{1} $ dus: $ \matdrie{ 1 & 0 &  2 }{ 0 & 1 & 1 }{0 & 0 & 1 } $. \\
We kijken wat het beeld van $ \hat{b}_{xa} = \vecdrie{1}{0}{1} $ onder \textit{$ T_a $} is: \\
$ T_a \cdot \hat{b}_{xa} \ 
= \ \matdrie{ 1 & 0 &  2 }{ 0 & 1 & 1 }{0 & 0 & 1 }  \cdot \vecdrie{1}{0}{1} \ 
=  \ \vecdrie{3}{1}{1} $  
\quad  Anders gezegd: \quad $ \vectwee{1}{0}  \  \xrightarrow{T}   \ \   \vectwee{ 3 } {1}. $\\ \\ \\
Wat is het beeld van de oorsprong onder \textit{$  T_a $}? Dan moeten we ook aan de oorsprong een dimensie toevoegen $ O_a = (0,0,1) $ Dan is \\ \\
$ T_a \cdot O_a \ 
= \ \matdrie{ 1 & 0 &  2 }{ 0 & 1 & 1 }{0 & 0 & 1 }  \ \cdot \ \vecdrie{0}{0}{1} \ 
=  \ \vecdrie{2}{1}{1} $ 
\quad dat wil zeggen \quad
$ \vectwee{0}{0}  \  \xrightarrow{T}   \   \vectwee{ 2 } {1}. $\\

\subsection{Een translatie in \RD}
Ook in 3 dimensies is een translatie niet lineair. Dus ook hier voegen we een dimensie toe:
\mydef []
{affiene matrix}{ Als $ \vec{t} = \vecdriesnel{t} $ \quad dan is  $ T_a = \matvier{ 1 & 0 & 0 & t_1 } 
	{ 0 & 1 & 0 &  t_2 }  
	{ 0 & 0 & 1 &  t_3 } 
	{0 & 0 &  0 & 1 } $ \quad de affiene matrix }\\
\mybv[affiene matrix]{Voor de translatie over $ \vec{t}=  \vecdrie{2}{3}{-4} $ \  is de affiene matrix $ T_a = \matvier{ 1 & 0 & 0 & 2} 
	{ 0 & 1 & 0 &  3}  
	{ 0 & 0 & 1 &  -4 } 
	{0 & 0 &  0 & 1 } $  }\\          
\mydef 
{affiene vector}
{ Als $\vec{v} = \ \vecdriesnel{v} $ \quad dan is \quad 
	$\vec{v}_a = \vecvier{v_1} {v_2} {v_3} {1}  $ \quad de affiene vector }\\
\mybv[affiene vector] 
{ Als $\vec{a} = \ \vecdrie{2}{-4}{-1} $ \quad dan is \quad 
	$\vec{a}_a = \vecvier{2} {-4} {-1} {1}  $ \quad de affiene vector }


\subsection{Rekenen met een affiene  matrix in \RD}
Nu kunnen we rekenen met een drie-dimensionale translatie.
Neem dezelfde translatie als hier vlak boven, over de vector $ \vec{t}=  \vecdrie{2}{3}{-4}. $ \\
Neem een willekeurige  vector $\vec{v} = \ \vecdriesnel{v} $\\  Dan is de affiene vector  $\vec{v}_a = \  \vecvier{v_1} {v_2} {v_3} {1} . $\\ 
Het beeld van $\vec{v} $ kunnen we nu uitrekenen: \\
$ T_a  . \ \vec{v}_a \ 
=   \matvier{ 1 & 0 & 0 & 2} 
{ 0 & 1 & 0 &  3}  
{ 0 & 0 & 1 &  -4 } 
{0 & 0 &  0 & 1 } \ \  \vecvier{v_1} {v_2} {v_3} {1} \ \ 
= \ \  \vecvier{v_1 + 2} {v_2 + 3 } {v_3 - 4 } {1} $ \\
dus 
\quad \quad \quad $ \vecdriesnel{v}  \  \xrightarrow{T}   \   \vecdrie{v_1+2} {v_2+3} {v_3-4} . $ 
\\ \\ \\
bijvoorbeeld 
$ \vecdrie{1}{1}{1}  \  \xrightarrow{T}   \   \vecdrie{3} {4} {-3} $ \ \quad en \
\quad $ \vecdrie{0}{1}{0}  \  \xrightarrow{T}   \   \vecdrie{2} {4} {-4} $ 

\section{Samenstelling}		
Bewegen bestaat natuurlijk niet alleen uit draaien (roteren), spiegelen enzovoort, maar vooral uit combinaties daarvan. Dus is het belangrijk dat we die combinaties ook uit kunnen rekenen (daarom was het belangrijk om ook matrices voor translaties te hebben). \\ 
\mydef []
{samenstelling}
{Als $ A $ en $ B $  matrices zijn dan is:\\
	de matrix $S$ van de afbeelding $A$ \textit{\textbf{gevolgd}} door $B$ gelijk aan:
	\quad $ S\ = \ B \cdot A $ \\ NB voor je gevoel draait de volgorde om! Maar als je goed kijkt zie je dat je eerst $A$ uitvoert en daarna $B$.}
\subsection{Samenstelling in \RT}
\mybv[samenstelling] 
{Stel dat de rotatie $ R 
	=    \mattwee { 1 & -1 }
	{ 1 & 1 } $ \quad en de projectie \quad 
	$ P = \  \mattwee { 0.5 & 0.5 }
	{ 0.5   & 0.5  } $ \\
	en we willen \textit{eerst} roteren en \textit{daarna} projecteren, \\ dan is de matrix van die samenstelling: \\
	$  S = P\cdot R 
	= \ \mattwee { 0.5 & 0.5 }
	{ 0.5 & 0.5 } \cdot 
	\mattwee { 1 & -1 }
	{ 1 & 1 } 
	= \ \mattwee { 1 & 0 }
	{ 1 & 0 }
	$ } \\
maar als we \textit{eerst} willen  projecteren en \textit{daarna}  roteren, \\ dan is de matrix van die samenstelling:\\
$  S_2 = R\cdot P
= \mattwee { 1 & -1 }
{ 1 & 1 } \cdot 
\ \mattwee { 0.5 & 0.5 }
{ 0.5 & 0.5 } 
= \ \mattwee { 0 & 0 }
{ 1 & 1 }
$ \\ \\ \\
\mybv[samenstelling] 
{Stel dat de rotatie $ R 
	=    \mattwee { \cos \theta & -\sin \theta }
	{ \sin \theta & \cos \theta} $ \quad en de projectie \quad 
	$ P = \ \frac{1}{17} \  \mattwee { 1 & 4 }
	{ 4  & 16 } $ }
en we willen \textit{eerst} projecteren  en \textit{daarna} roteren, \\
dan is de matrix van die samenstelling: 
\begin{align*} 
S = R\cdot P 
&=	  \mattwee { \cos \theta & -\sin \theta }
{ \sin \theta & \cos \theta}  \ \cdot  \ 
\frac{1}{17} \  \mattwee { 1 & 4 }
{ 4  & 16 }  \\
& = \   \frac{1}{17} \  
\mattwee { 1\cdot \cos \theta + 4\cdot -\sin \theta  & & 4\cdot \cos \theta + 16\cdot -\sin \theta  }
{ 1\cdot \sin \theta + 4\cdot \cos \theta &  &4\cdot \sin \theta + 16\cdot \cos \theta}  \\  
& = \   \frac{1}{17} \  
\mattwee { \cos \theta - 4\sin \theta  & & 4\cos \theta - 16\sin \theta  }
{ \sin \theta + 4\cos \theta & & 4\sin \theta + 16\cos \theta}                                                
\end{align*} 

\subsection{Samenstelling met affiene matrices in \RT}
Als we een translatie in \RT willen samenstellen met een andere afbeelding moeten we eerst aan alle afbeeldingen een dimensie toevoegen. \\ \\
\mydef[Als
$  A = \mattwee{a_{11} & a_{12} } 
{ a_{21} & a_{22} } $]
{affiene matrix}{
	dan is de affiene matrix daarvan, 
	$  A_a = \matdrie{a_{11} & a_{12}  & 0 } 
	{ a_{21} & a_{22}  & 0 }
	{0 & 0 & 1 }  $  }  \\ \\ 
\mybv[affiene samenst.]
{ Stel $T$ is de translatie over $\vec{t} = \ \vectwee{4}{-3} $ en dat we daarnaast willen roteren met \  \   
	$ R  =  \mattwee { 1 & -1 }
	{ 1 & 1 } $  \ en willen met 
	$ S  =  \mattwee { 0 & 1 }
	{ 1 & 0}.  $ Dan zijn de affiene matrices: \\}
$ T_a = \matdrie{ 1 & 0 & 4 }
{ 0 & 1 & -3 }
{0 & 0 & 1 } $  \qquad
$ R_a = \matdrie{ 1 & -1 & 0 }
{ 1 & 1 & 0 }
{0 & 0 & 1 } $ \qquad
$ S_a = \matdrie{ 0 & 1 & 0 }  
{ 1 & 0 & 0 }
{0 & 0 & 1 } $ \\ \\ \\
Dan is de samenstelling van de drie, bij eerst spiegelen, dan transleren en tot slot roteren:
\begin{align*} 
R_a . T_a . S_a & = \ 
\matdrie{ 1 & -1 & 0 }
{ 1 & 1 & 0 }
{0 & 0 & 1 } \ . \ 
\matdrie{ 1 & 0 & 4 } 
{ 0 & 1 & -3 }
{0 & 0 & 1 } \ . \ 
\matdrie{ 0 & 1 & 0 }  
{ 1 & 0 & 0 }
{0 & 0 & 1 }   \\
& = \ 
\matdrie{ 1 & -1 & 0 }
{ 1 & 1 & 0 }
{0 & 0 & 1 } \ . \ 
\matdrie{ 0 & 1 & 4 }  
{ 1 & 0 & -3 }
{0 & 0 & 1 }   \\
& = \ 
\matdrie{ -1 & 1 & 7 }  
{ 1 & 1 & 1 }
{0 & 0 & 1 }   
\end{align*} 

\newpage
\subsection{Samenstelling met affiene matrices in \RD}
Als we een translatie in \RD willen samenstellen met een andere afbeelding moeten we, net als in \RT eerst aan alle afbeeldingen een dimensie toevoegen. \\ 
\mydef[]
{affiene matrix} {
	Als $ 	A = \matdrie {{}a_{11} & a_{12} & a_{13} } 
	{ a_{21} & a_{22}  &  a_{23}}
	{ a_{31} & a_{32} & a_{33}} $ \ \ 
	dan is \ 
	$  
	A_a = \matvier{a_{11} & a_{12} & a_{13} & 0 } 
	{ a_{21} & a_{22}  & a_{23} & 0 }
	{ a_{31} & a_{32} & a_{33} & 0}
	{0 & 0 & 0 & 1 }  $ \ de affiene matrix van $A$. } 
\mybv[affiene samenst.] 
{ Stel $T$ is de translatie over $\vec{t} = \ \vecdrie{4}{-3}{2}, $ \  en    
	$ R  =  \matdrie { 1 & -1 & 0}
	{ 1 & 1 & -1 } 
	{ 0 & -1 & 1 }$ , \ en 
	$ P  =  \matdrie { 0 & 1  & 0 }
	{ 1 & 0 & 0}
	{0 & 0 & 1}.  $  \\ Dan zijn de affiene matrices: }
$ T_a = \matvier{ 1 & 0 & 0 & 4 }
{ 0 & 1 & 0 &-3 }
{ 0 & 0 & 1 & 2}
{0 & 0 & 0 & 1 } $  \qquad
$ R_a = \matvier{ 1 & -1 & 0 & 0 }
{ 1 & 1 & -1 & 0}
{ 0 & -1 & 1  & 0}
{0 & 0 & 0 & 1 } $ \qquad
$ P_a = \matvier{ 0 & 1 & 0  & 0 }  
{ 1 & 0 & 0 & 0 }
{ 0 & 0 & 1  & 0 } 
{0 & 0 & 0 & 1 } $ \\ \\
en dan is de samenstelling van eerst $P$, dan $T$ en tot slot $R$:
\begin{align*} 
R_a \cdot  T_a \cdot  P_a & = \ 
\matvier{ 1 & -1 & 0 & 0 }
{ 1 & 1 & -1 & 0}
{ 0 & -1 & 1  & 0}
{0 & 0 & 0 & 1 }  \ \cdot  \ 
\matvier{ 1 & 0 & 0 & 4 }
{ 0 & 1 & 0 &-3 }
{ 0 & 0 & 1 & 2}
{0 & 0 & 0 & 1 }  \ \cdot  \ 
\matvier{ 0 & 1 & 0  & 0 }  
{ 1 & 0 & 0 & 0 }
{ 0 & 0 & 1  & 0 } 
{0 & 0 & 0 & 1 }  \\
& = \ \matvier { 1 & -1 & 0 & 0 }
{ 1 & 1 & -1 & 0}
{ 0 & -1 & 1  & 0}
{0 & 0 & 0 & 1 }  \ \cdot  \ 
\matvier{ 0 & 1 & 0 & 4 }
{ 1 & 0 & 0 &-3 }
{ 0 & 0 & 1 & 2}
{0 & 0 & 0 & 1 }   \\
& = \ \matvier{ -1 & 1 & 0 & 7 }
{ 1 & 1 & -1 & -1 }
{ -1 & 0 & 1 & 5}
{0 & 0 & 0 & 1 }  
\end{align*} 
Dit alles betekent tot slot dat als $\vec{v} = \vecdriesnel{v} $ een willekeurige vector in \RD is, \\dan is  het beeld van $ \vec{v} $ onder $  R_a \cdot  T_a \cdot  P_a $ :
\begin{align*} 
\vec{v}  \  \xrightarrow{R_a \cdot  T_a \cdot  P_a}   
&   \matvier{ -1 & 1 & 0 & 7 }
{ 1 & 1 & -1 & -1 }
{ -1 & 0 & 1 & 5}
{0 & 0 & 0 & 1 }  \ \cdot  \ 
\matvier{v_1}{v_2}{v_3}{1}  \\ 
=  & \matvier {-v_1+v_2+7} 
{ v_1+v_2-v_3 -1 }
{-v_1+v_3+ 5} {1} 
\end{align*}                        
Dus bijvoorbeeld: \\
$ \vecdrie{1}{1}{1}  \  \xrightarrow{R\cdot  T\cdot  P} \  \vecdrie{7}{0}{5}    $ \quad en  \quad 
$ \vecdrie{1}{0}{0}  \  \xrightarrow{R\cdot T\cdot P}  \ \vecdrie{6}{0}{4}  $ 
\quad en  \quad 
$ \vecdrie{0}{0}{1}  \  \xrightarrow{R\cdot T\cdot P}  \ \vecdrie{7}{-2}{6}  $ 

\newpage
\section{Opgaven}
\begin{enumerate}
	\item Geef de affiene matrix van de translatie $ T $  die punt (3,2) afbeeldt op (5,-2).	\\   
	
	\item Geef de matrix van de spiegeling  $ S_1$ in \RD in de lijn $ x = 3 y $ \\
	
	\item Gegeven de translatie $ T_{3} $ die punt (1,2) afbeeldt op punt (3,-2) en de rotatie \\
	$ R  =  \mattwee  {  \cos 62\degree & \sin 62\degree}
	{ -\sin 62\degree & \cos 62\degree} $. \ \\ \\
	Geef de matrix van de samengestelde afbeelding $B$ die bestaat uit $R$ gevolgd door $ T_{3} $. NB: je moet eerst $ T_{3} $ en $R$ affien maken! \\
	
	\item Gegeven de translatie $ T_{4} $  die punt (-1,0,3) afbeeldt op punt (2,-1,4) en de spiegeling 
	$ S = \frac{1}{6} \matdrie{ 2& -1 &  2}{ 0& -1 & 2}{ -2 & -3 & -1}. $  \ 
	Geef de matrix van de afbeelding $C$ die bestaat uit de translatie gevolgd door de spiegeling.
\end{enumerate} 
\vspace{2cm}

\subsection{Extra opgaven}
\begin{enumerate}
	\item Geef de matrix van de spiegeling $ S_2 $  in \RD in het vlak $ y= 2x $ \\
	
	\item Geef de matrix van de spiegeling $ S_3 $  in \RD in het vlak $ z= -\frac{1}{3}y $ \\
	
	\item Geef de affiene matrix van de translatie $ T_{1} $ die punt (5,1,2) afbeeldt op (0,1,6). \\
	
	\item Geef de affiene matrix van de translatie $ T_{2} $ die punt (9,0,-1) afbeeldt op (6,2,-2). \\
	
	\item Geef de matrix van de afbeelding $A$ die samengesteld is uit $ S_2 $, gevolgd door $ T_1 $, gevolgd door $ S_3 $.  
\end{enumerate}

