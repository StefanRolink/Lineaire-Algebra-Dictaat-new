% MACORO'S
% \newcommand{\mydef}[3] []{
% #1\index[definities]{#2}     \\
%   \marginnote{\textbf{\Large $ \Delta $}
%     #2}[0cm] \quad #3   \\ \\}
% \newcommand{\figuur}[3][0.4]{
% \begin{figure}[h]
%     \centering
%     \includegraphics[width=#1\linewidth]{figuren/#2}
%     \caption{#3}
%     \label{fig:#2}
%     \index[figuren]{#3}
% \end{figure}
% \FloatBarrier	}
% \newcommand{\mycent}[1]{\begin{center} #1\end{center}}
% \newcommand{\mybv} [2][]{
% \index [voorbeelden]{#1}
% \marginnote{ {\LARGE $\nu$} #1}[0cm]
% #2\\
% }
% \newcommand{\myeig} [2][]{
% \index [eigenschappen]{#1}
% \marginnote{ \textit{{\LARGE $\epsilon$}} #1}[0cm]
% #2\\
% }



\chapter{Algemene Informatie}
\label{chap:algemeneinformatie}

\subsubsection{Doelgroep en instroomeisen}
Dit dictaat is bedoeld voor 2e jaars studenten HBO-ICT die de specialisatie SE hebben gekozen. Voor deze module gelden geen instroomeisen. Maar het is wel erg handig als je de wiskunde uit het eerste jaar onder de knie hebt.

\subsubsection*{Leeruitkomsten}
De leeruitkomsten van dit vak zijn dat je
\begin{itemize}
        \setlength\itemsep{-1pt}
        \item kunt rekenen met vectoren, matrices, lijnen en vlakken  in \RT en \RD
        \item matrices van afbeeldingen in \RT en \RD kunt bepalen
        \item een determinant kunt uitrekenen
        \item kunt rekenen met quaternionen
        \item begrip hebt van de mogelijkheden die bovenstaande zaken bieden voor grafische applicaties
        \newline
    \end{itemize}

    


\subsubsection{Symbolen}
In de wiskunde wordt veel gebruik gemaakt van (misschien onbekende) symbolen. Het is belangrijk om daar precies mee om te gaan. Bijvoorbeeld: $ \vec{a} \ne \hat{a} $, wat in normale mensentaal betekent: de \textit{vector a} is iets anders dan de \textit{eenheidsvector a}. 
De meeste symbolen in dit dictaat worden algemeen gebruikt in de wiskunde maar er zijn drie speciale alleen voor dit dictaat:\\
\marginnote{\textbf{{\Large $ \Delta $ } } \ Definitie }[0cm]
Met het teken {\Large $ \Delta $ }in de kantlijn wordt een definitie aangegeven.\\
\marginnote{\textbf{{\LARGE $ \nu $ } } Voorbeeld }[0cm]
Met het teken {\LARGE $ \nu $ }  in de kantlijn wordt een voorbeeld aangegeven.\\
\marginnote{\textbf{{\LARGE $ \epsilon $ } } \ Eigenschap }[0cm]
Met het teken {\LARGE $ \epsilon $ }  in de kantlijn wordt  een eigenschap aangegeven.\\

Zoals normaal in de wiskunde gebruiken we letters uit het Griekse alfabet, bijvoorbeeld $ \alpha $ (alfa, de Griekse a) de letter die het alfabet z'n naam gegeven heeft, en  $ \lambda $ (lambda, de Griekse l)\\
En toch zijn soms zelfs wiskundigen slordig. Bijvoorbeeld bij het verschil tussen punten en vectoren. Een punt is heel iets anders dan een vector  (zie figuur \ref{fig:vectorVB3}), en toch zullen de termen door elkaar gebruikt worden zolang er geen misverstand mogelijk is.

\subsubsection{Wiskunde lezen}
Bij wiskundige teksten gaat het lezen veel langzamer  dan bij 'gewone' teksten. Het is niet raar als je een zin 3 keer moet lezen voor dat je hem snapt. (en dan nog kan het zijn dat je eerst de voorgaande zin nog een keer moet lezen). Kortom lezen en begrijpen van wiskunde kost tijd en oefening. \\

In de wiskunde is het gebruikelijk om van alles wat je opschrijft ook het bewijs te leveren. Dat doen we in dit dictaat niet, omdat dat te veel zou afleiden, wel wordt zoveel mogelijk uitleg gegeven. Wil je nagaan of de dingen die hier instaan kloppen, dan kun je of het zelf proberen te bewijzen, of opzoeken op internet.

\subsubsection{Leeswijzer}
Aan het einde van elk hoofdstuk staan opgaven èn 'extra opgaven'. Het idee is dat de gewone opgaven in principe genoeg oefening bieden om de stof te beheersen. Wil je toch nog meer oefenen of herhalen dan zijn daarvoor de extra opgaven. In hoofdstuk 2 wordt de basis van vectoren behandeld. Hoofdstuk 3, 4 en 5 gaan over matrices, de belangrijkste hulpmiddelen voor maken van grafische berekeningen, die in alle games en (bewegende) 3D-zaken gebruikt worden. Hoofdstuk 3 gaat over draaiingen en projectie, hoofdstuk 4 over spiegelen, translaties en samenstellingen. Hoofdstuk 5 gaat over een belangrijke eigenschap van matrices: determinanten. Hoofdstuk 6 gaat over een manier van rekenen, quaternionen, met behulp van complexe getallen, om de zogeheten 'Gimbal Lock' te vermijden. En in Hoofdstuk 7 tenslotte staan (korte) antwoorden op de opgaven bij elk hoofdstuk. \\

Dit dictaat is geschreven in \LaTeX\ (spreek uit: lateg) met behulp van het programma Neovim. In tegenstelling tot Word biedt  \LaTeX \   heel veel opmaak mogelijkheden.

