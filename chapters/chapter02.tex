\chapter{Vector}
\label{chap:vectoren}
In dit hoofdstuk behandelen we vectoren en hoe vectoren in de wiskunde gebruikt worden.\\ \\

\mydef[Een \textit{vector} is ] {vector} {een verzameling getallen die in een rij of kolom gerangschikt zijn.}
In dit dictaat bestaat een vector alleen uit reële getallen (dwz gehele getallen, breuken en getallen als $  \pi $ of $ \sqrt{2} $ ). Alle reële getallen samen noemen we \rm I\!R, de twee-dimensionale ruimte met reële getallen \RT, en de drie-dimensionale ruimte \RD.
Een vector wordt hier geschreven als een kleine letter met een pijltje erboven, bijvoorbeeld $\vec{b}$ en: \\

\mybv[vector \RD, \RT]{  \qquad  
	$\vec{x} = \vecdriesnel {x}   
	\quad    \quad
	\vec{v} = \vectwee {x}{y} 
	\quad    \quad 
	\vec{s} = \vecdrie {-3}{2}{4}  
	\quad    \quad 
	\vec{a} = \vectwee {1}{2} $ }

$\vec{x}$ is een  vector in ${\rm I\!R^{3}}$,  een  vector in de drie-dimensionale ruimte, $ \vec{x} $  bestaat uit 3 getallen: $  x_{1},  x_{2} $  en  $ x_{3}$ .  $\vec{v}$  is een  vector in ${\rm I\!R^{2}}$, een twee-dimensionale vector in het platte vlak, $\vec{s}$ een concrete vector in  ${\rm I\!R^{3}}$ en $ \vec{a}  $ een concrete vector in \RT (zie figuur  \ref{fig:vectorVB3}).

\figuur[0.9]{vectorVB3}{(a) De \textit{vector} $ \vec{a}  $ in \RT is het 'pijltje' vanuit O, de oorsprong, naar het punt (1,2) met een lengte en een richting. Het \textit{punt P(2,1)} heeft  \textit{geen} lengte of richting. (b) De vector $  \vec{s} $ in \RD loopt iets naar achter omhoog}


\section{Rekenen met vectoren}
Je kunt op verschillende manieren rekenen met vectoren.
\subsubsection{Optellen}
Het optellen van vectoren doe je door de overeenkomstige elementen van de vectoren bij elkaar op te tellen. Als volgt:\\

\mydef[De som van twee vectoren  \V{a}  en  \V{b} \  is:\\] {som}
{\V{a}  +  \V{b} \  $ = \  \vecdriesnel {a}  \  +  \  \vecdriesnel {b} \   = \  \vecdrie {a_1+b_1}  {a_2+b_2}  {a_3+b_3}  $ \\
	Je kunt 2 vectoren alleen maar bij elkaar optellen als ze dezelfde dimensie hebben, dat wil zeggen een gelijk aantal elementen.} \\ 

\mybv[som \RD]{
	$\quad \vecdrie {3}{-1}{2}  \  +  \  \vecdrie {0}{6}{-4}  \  =  
	\ \vecdrie {3+0}  {-1+6}  {2+-4}  \ =
	\ \vecdrie {3}  {5}  {-2} $ }\\

\mybv[som \RT]{ 
	$\quad \vectwee {-3}{6}  \ +  \  \vectwee {7}{2}  \  =  
	\ \vectwee {-3+7}   {6+2}  \  =
	\ \vectwee {4}  {8}   $  \qquad (zie figuur  \ref{fig:som2}).}

\figuur[0.3]{som2}{De som van de vectoren $  \vec{a}  $       en $  \vec{b}   $ is weer een vector }

\subsubsection{Scalair product}
Als je een vector vermenigvuldigt met een getal heet dat vermenigvuldigen met een scalar (scalair product). Bij scalaire vermenigvuldiging worden alle elementen van de vector met datzelfde getal vermenigvuldigd.\\

\mydef[Het scalair product van een getal c en een vector  \V{a}  \ is:]
{scalair  product} 
{ \  $ c . \vec{a}  \   = \ c.    \vecdriesnel {a}  \  =  \ \vecdrie {c.a_1}  {c.a_2}  {c.a_3} $\\ \\}

\mybv[scalair product]
{Als $  \vec{a} =  \vectwee{-4}{-3} $  en $c= -2$ \quad dan is $ -2 . \vec{a} = -2 . \vectwee{-4}{-3}  \    
	= \ \vectwee {-2.-4}{-2.-3}  \  
	=  \ \vectwee  {8}  {6}. $  \quad (figuur  \ref{fig:scalair}) }

\figuur[0.4]{scalair}{Het scalair product van -2 en $  \vec{a}  $ }

\subsubsection{Verschil}
Vectoren van elkaar aftrekken doe je door de overeenkomstige elementen van elkaar af te trekken:\\

\mydef[Het verschil tussen twee vectoren  \V{a}  en  \V{b} \ is:]{verschil}
{ \V{a} \ - \ \V{b} \  \   $ = \  \ \vecdriesnel {a}  \  -  \ \  \vecdriesnel {b}  \ \  = \  \ \vecdrie {a_1-b_1}  {a_2-b_2}  {a_3-b_3} $\\}

\mybv[verschil \RD]{
	$\vecdrie {3}{-1}{2}  \ -  \ \vecdrie {0}{6}{-4}  \  =  
	\ \vecdrie {3-0}  {-1-6}  {2--4}  \  =
	\ \vecdrie {3}  {-7}  {6} $ }\\ \\

\mybv[verschil \RT]{
	$\vectwee {-3}{6}  \ - \  \vectwee {7}{2}  \  =  
	\ \vectwee {-3 - 7}  {6-2}   \  =
	\ \vectwee {-10}  {4}  $ \quad (figuur \ref{fig:verschil2})}\\

Net zo als bij het  optellen van vectoren gedlt bij het aftrekken dat alleen vectoren met dezelfde dimensie van elkaar kunnen worden afgetrokken. 
Zoals je in figuur \ref{fig:verschil2} kunt zien is het verschil van twee vectoren $  \vec{a}  $  en $  \vec{b}  $ hetzelfde als het optellen van de vectoren $  \vec{a}  $  en $ - \vec{b}  $. (vergelijk met figuur  \ref{fig:som2} )

\figuur[0.5]{verschil2}{Het verschil van de vectoren $  \vec{a}  $  en $  \vec{b}  $ is $ \ 
	\vec{a}\  - \ \vec{b} \  = \   \vec{a} \  + \ -\  \vec{b}$}

\subsubsection{Inproduct}
Je kunt niet zomaar 2 vectoren met elkaar vermenigvuldigen. Maar er zijn toch twee manieren om iets te doen wat er op lijkt. 
Het inproduct van twee vectoren is een getal(!), geen vector. We noteren het inproduct als: $(\vec{a}, \vec{b}) $ en we berekenen het als volgt: (We gebruiken hier stippeltjes . . .  en het subscript $_n$ om aan te geven dat het over òf 2 òf 3 òf nog meer dimensies kan gaan)\\ \\

\mydef[Het inproduct van \V{a}  en  \V{b} \  is:]{inproduct}
{ \  \quad $ (\vec{a}, \vec{b}) = a_1b_1 + a_2b_2 + . . . + a_nb_n$\\}

\mybv[inproduct]{ als  $\vec{a} = \vecdrie {3}{-1}{2}  \ $ en  \ \  $\vec{b} = \vecdrie {0}{6}{-4}  \ $\\ \\
	dan is $ (\vec{a}, \vec{b}) = 3.0 +-1.6 +2.-4 = 0-6-8=-14$}


\subsubsection{Loodrechte vectoren}
Twee vectoren staan loodrecht op elkaar als de hoek tussen beide vectoren $ 90^{\circ} $ is. Dat leidt de volgende belangrijke eigenschap van het inproduct: \\ \\

\myeig[inproduct]
{
	$(\vec{a}, \vec{b}) = 0   \quad      dan\  en \ slechts\  dan \ als \quad	\vec{a} $  loodrecht op $ \vec{b} $    
	\quad \quad	($ \vec{a} \ne \vec{0} $ en   $ \vec{b} \ne \vec{0} $).
}

\figuur[0.3]{loodrecht}{de vectoren $  \vec{a}  $  en $  \vec{b}  $ staan loodrecht op elkaar}

\mybv [loodrecht] 
{Stel dat $ \vec{a} = \vectwee{1}{-2} $ en $ \   \vec{b} = \vectwee{2}{1} $ \ dan is $ (\vec{a}, \vec{b})  = 1.2 + -2.1 = 0 $ \ \ Zie figuur  \ref{fig:loodrecht} } 

\subsubsection{Uitproduct}
De andere manier waarop je 2 vectoren kunt 'vermenigvuldigen' heet het uitproduct of kruisproduct. Het uitproduct van 2 vectoren levert weer een vector op. Het uitproduct is alleen maar gedefini"eerd in \RD.\\

\mydef[Het uitproduct van \V{a}  en  \V{b} \  is:]{uitproduct}
{$\vec{a} \times \vec{b} \ = \ \vecdriesnel{a}  \ \times  \ \vecdriesnel{b} \ 
	= \ \vecdrie{a_2.b_3 -a_3. b_2}{-(a_1.b_3 -a_3.b_1)}{a_1.b_2 - a_2.b_1} $} \\

\mybv[uitproduct]{ als $ \vec{a} =  \ \vecdrie{-1}{2}{4} $ en  $ \vec{b} =  \ \vecdrie{3}{-2}{1} $ 
	\quad dan is \quad $ \vec{a}  \times  \vec{b} \ = \ \vecdrie{2.1 - 4.-2}{-(-1.1-4.3)}{-1.-2 - 2.3} = \ \vecdrie{10}{13}{-4} $}\\ \\

\myeig[uitproduct]
{ Een belangrijke eigenschap van het uitproduct is de volgende:\\
	De vector $ \vec{c} \ = \ \vec{a} \times \vec{b} $ staat loodrecht op  $ \vec{a} $ en loodrecht op  $ \vec{b} $}\\

\mybv[loodrecht]{ Neem $ \vec{a}, \   \vec{b} $   en   $ \vec{c} $ als hierboven.
	dan is  $(\vec{c}, \vec{a}) = 10.-1 + 13.2 -4.4 = -10+26-16 =0 $ \\
	en $(\vec{c}, \vec{b}) = 10.3 + 13.-2 -4.1 = 30-26-4=0 $}

\figuur[0.4]{loodrecht3d}{de vector $  \vec{c}  $  staat loodrecht op $  \vec{a}  $  \textbf{\textit{en}} $  \vec{b} $}

\section{Vectoren en meetkunde}
Er bestaan een aantal meetkundige operaties die we op vectoren kunnen toepassen.
Om  te beginnen, de norm, of lengte van een vector.

\subsubsection{Norm}
De norm van een vector is de lengte daarvan. Dat is in \RD en \RT makkelijk  voor te stellen: je neemt gewoon de "lengte van het pijltje". De norm wordt geschreven met 2 verticale streepjes:\\

\mydef[De norm van een vector \V{a} is:\\ ]{norm}
{$|\vec{a}| = \sqrt{ {a_1}^{2} + {a_2}^{2} + . . . +{a_n}^{2}}$\\ \\}

\mybv[norm]{
	als \ $ \vec{v} = 	\vecvijf{3}{1}{-3}{4}{-1} \  $ 
	dan is  $|\vec{v}| = \sqrt{ 3^{2} + 1^{2} +(-3)^{2} + 4^{2}  +(-1)^{2}} = 6$}

\subsubsection{Eenheidsvector}
Soms is het nodig om vectoren met lengte 1 te hebben:
\mydef[]
{eenheidsvector}
{Een eenheidsvector is een vector met lengte 1.}
Een eenheidsvector wordt genoteerd met een accent circonflexe (\textasciicircum \ , dakje):\\ 

\mybv[eenheidsvector]
{\qquad  $\hat{e} = \vecdrie {0}{0}{1} $}\\ 
Een willekeurige vector $ \vec{v} $  is te schalen naar een eenheidsvector door $ \vec{v} $   te vermenigvuldigen met $ \frac{1}{norm} =  \frac{1}{ |\vec{v}|} $. 
Dat wil zeggen $\hat{v} =  \frac{1}{ |\vec{v}|}.\vec{v} $
De berekening van een eenheidsvector in 5 dimensies gaat als volgt: \\ \\

\mybv[eenheidsvector]{als \ $ \vec{v} =\  \vecvijf{3}{1}{-3}{4}{-1} $, \ dan is $  |\vec{v}| = 6$ (zie boven) 
	en is $ \hat{v} =  \frac{1}{ |\vec{v}|}.\vec{v} = \frac{1}{6} .\vecvijf{3}{1}{-3}{4}{-1} \ = \  \vecvijf{\nicefrac{3}{6}}{\nicefrac{1}{6} }{-\nicefrac{3}{6}}{\nicefrac{4}{6}}{-\nicefrac{1}{6}}\ \  = \  \vecvijf{\nicefrac{1}{2}}{\nicefrac{1}{6}}{-\nicefrac{1}{2}}{\nicefrac{2}{3}}{-\nicefrac{1}{6}} $}

\subsubsection{Hoek tussen twee vectoren}
We kunnen het inproduct gebruiken om de hoek tussen twee vectoren te berekenen. 
\mydef[De formule voor de hoek tussen  vectoren \V{a}  en  \V{b} \  gebruikt het inproduct en  de norm:]{hoek}
{\quad  
	$\cos \alpha = \dfrac{(\vec{a}, \vec{b}) }{|\vec{a}| |\vec{b}|} $}
Daarbij is $\alpha \ $ (\textit{alfa}) de hoek tussen de vectoren $\vec{a}$ en $\vec{b}.$ En cos staat voor cosinus, een maat voor de hoek, die je eenvoudig met je rekenmachine kunt uitrekenen. Zie figuur \ref{fig:hoek}  \\ \\

\figuur[0.3]{hoek}{Met $ \alpha $ geven we de hoek tussen twee vectoren aan}

\mybv[hoek]{De berekening van een  hoek tussen twee vectoren gaat als volgt:\\
	Stel   
	$ \vec{a} = \vecdrie {2}  {-2} {1}  $ \ en \  $ \vec{b} = \vecdrie {-3}  {0} {4}  $ 
	\qquad dan is 
	\qquad $ (\vec{a}, \vec{b}) = 2.-3 + -2.0  + 1.4 = -2 $.\\ 
	De lengte van de vectoren is: 
	$ |\vec{a}| = \sqrt{9} = 3, \ |\vec{b}| = \sqrt{25}=5 $. \\
	Invullen in de formule levert: 
	$\cos \alpha = \dfrac{-2}{3.5} = - \dfrac{2}{15}. $  \ Met de rekenmachine berekenen we het aantal graden van de hoek: 
	$\alpha \ = \cos ^{-1}(- \dfrac{2}{15}) \ \approx \ ~ 97,7 ^{\circ} $ } 

\section{Vectorvoorstelling van  lijn en  vlak}
Uit de lessen Wiskunde Basis is bekend dat de vergelijking van een lijn in \RT in het algemeen $ y = ax + b $ is.  Daarnaast bestaan er vectorvoorstellingen en vergelijkingen van lijnen en vlakken. 

\subsection{Definities lijn}
\mydef [De \textit{vergelijking} van een lijn $ l $ in \RT is: ] 
{lijn \RT}  {$ y = ax + b $ 
	\qquad \qquad \quad \quad met \textit{a} de richtingscoëffici"ent  en \textit{b} een constante.}

\mybv[lijn \RT] 
{$l: y = -\frac{1}{2}x +4. $   \ \ De richtingscoëfficiënt  van \textit{ l }   is $ - \ \frac{1}{2}. $ ( figuur  \ref{fig:lijn2d})
}\\ \\

\mydef  [De \textit{vectorvoorstellling} van een lijn $ l $ in \RT is: ]
{lijn \RT}  
{ $  \vectweesnel{x} \  = \ \vectweesnel{b} \  + \  \lambda \ \vectweesnel{r} $ 
	\qquad  \qquad of: 
	$ \vec{x}  =   \vec{b}  + \  \lambda \ \vec{r} $} 

Met  $ \vec{x}  =  \vectweesnel{x} $ en 
\ beginvector  $ \vec{b} =  \vectweesnel{b} $ ,  
\  richtingsvector $ \vec{r} =  \vectweesnel{r} $  \ en
$ \lambda $  (lambda) een variabel getal. Let op het verschil tussen een rcihtings\textit{vector} en een richtings\textit{coëfficiënt}.\\ 

\mybv[lijn \RT]
{Een voorbeeld van een vectorvoorstellling is:  
	$ m: \vectweesnel{x} \ = \ \vectwee{-5}{0} \  + \ \lambda \ \vectwee{0}{-1}. $\\  Zie figuur  \ref{fig:lijn2d}. 
	De richtingsvector van m is 
	$  \overrightarrow{rv_m} =\vectwee{0}{-1} \ $ 
	en de beginvector 
	$  \vec{b} = \vectwee{-5}{0} \ $
} 
\figuur[0.5]{lijn2d}{de lijn  $ l: y = -\frac{1}{2}x +4. $ met  $ r.c.= -\frac{1}{2}$ en de lijn \textit{m} met de rode richtingsvector en zwarte beginvector.}

\subsection{Definities vlak}
\mydef [De \textit{vergelijking} van een vlak $ V $ in \RD is: ]
{vlak \RD} 
{$ ax_1 + bx_2 + cx_3 = d $. \quad  \quad (a, b, c en d zijn constanten) } \\

\mybv[vlak \RD]
{Een voorbeeld van een vergelijking van een vlak $ V $ is: $ 2x_1 - 3x_2 +7x_3 = -5$ }\\  \\

\mydef [De \textit{vectorvoorstellling} van een vlak $ V $ in \RD is: ]
{vlak \RD}  
{  $ \vecdriesnel{x} \ =\ \vecdriesnel{b}  \ +  \lambda. \vecdriesnel{v} \   + \  \mu. \vecdriesnel{w} $ \qquad \qquad  of:  $ \vec{x}  =   \vec{b}  + \  \lambda \ \vec{v} + \  \mu \ \vec{w}  $. }\\ 
Met $ \vec{x}  = \   \vecdriesnel{x}  \quad    \vec{b} = \  \vecdriesnel{b} \quad    \vec{v} =  \  \vecdriesnel{v} \quad   \vec{w} =  \ \vecdriesnel{w} $  \\   $ \lambda $ en  $ \mu $ zijn variabele getallen ( $ \mu $, spreek uit: muu, is de Griekse letter m). En  $ \vec{b} $ is een constante beginvector en $ \vec{v}  $  en $ \vec{w}  $ zijn richtingsvectoren van vlak \textit{V}.\\ \\ \\ 

\mybv[vlak \RD]
{Een voorbeeld van een vectorvoorstellling van een vlak $ V $ is: \\
	$ \qquad	
	V: \   \vecdriesnel{x} \ = \ \vecdrie{1}{-3}{0} \ + 
	\lambda.  \vecdrie{1}{2}{-2} \   + 
	\  \mu. \vecdrie{-4}{1}{2} $ }

Wat  $ \lambda $ (spreek uit: lambda) en  $ \mu $ (spreek uit: muu) zijn wordt verderop bij de berekeningen uitgelegd (zie ook figuur  \ref{fig:vlak}.).  In lineaire algebra heb je soms de vergelijking, soms de vectorvoorstelling nodig. In de rest van deze paragraaf wordt uitgelegd hoe je van een vergelijking een verctorvoorstelling maakt en omgekeerd, zowel voor een lijn in \RT \ als een vlak in \RD. 

\figuur[1]{vlak}{Het vlak $ V $ heeft de beginvector $  \vec{s} $, en richtingsvectoren $  \vec{v} $ en $  \vec{w} $. Het punt P ligt op V en kun je vanuit \textit{\textbf{O}}, de oorsprong bereiken door  $  \vec{s} + \lambda. \vec{v} + \mu.\vec{w} $ met \ $  \lambda = 2 $ en$ \  \mu = 1,5 $. Elk punt van \textit{V} is op zo'n manier te bereiken. Met andere woorden $  V: \vec{x} =   \vec{s} \ + \  \lambda.\vec{v} \  +\  \mu.\vec{w}  $  }

\subsection{Berekeningen lijn}

\subsubsection{Van vectorvoorstelling lijn naar vergelijking lijn:} 
Een voorbeeld van een  vectorvoorstelling van een lijn $ l $ in \RT  is:
$ l: \ \vectwee {x}{y} = \vectwee {1}{2} + \lambda  \vectwee {2}{-1}  $.   (zie figuur \ref{fig:lijnVV} ) Dat betekent dat de lijn\textit{ l} door het punt $ (1,2) $ gaat en als richtingsvector de vecotr $ \vectwee {2}{-1} $ heeft. $ \lambda $  is een parameter, dat wil zeggen dat om een punt op de lijn $ l $ te vinden mogen we een waarde voor $ \lambda $  kiezen (2, of 100, of $ -\frac{2}{5} $, of ...). Bijvoorbeeld bij  $ \lambda= 3 $ vinden we dat het punt  (7,-1) = (1 + 3.2 , 2 + 3.-1)   op $ l $ ligt. Hoe maken we een vergelijking van  $ l $? In een vergelijking komt geen $ \lambda $ voor, dus moeten we zorgen de $ \lambda $ uit de vectorvoorstelling "kwijt te raken". Als je goed naar de vectorvoorstelling van $ l $ kijkt, zie je dat het eigenlijk 2 vergelijkingen zijn, één voor de x-coördinaat en één voor de y-coördinaat:

\[\begin{cases}
x = 1 + 2\lambda\\
y = 2 - \lambda 
\end{cases}
\] 
Dit is een stelsel van 2 vergelijkingen  en kunnen we 'oplossen' door $ \lambda $ te elimineren ('weg te werken'):
Uit de $ 2^{e}$ vergelijking $ y = 2 - \lambda $ volgt  dat $\lambda = 2 - y $. Invullen van  $ \lambda $ in de $ 1^{e}$ vergelijking levert $ x = 1 +2(2-y) $. Dan is $ x= 5 -2y $, oftewel $ l:  \ y = - \frac{1}{2}x + \frac{5}{2} $,  wat dus dezelfde lijn is als: $ l: \ \vectwee {x}{y} = \vectwee {1}{2} + \lambda  \vectwee {2}{-1}  $, waar we mee begonnen, zie de tekening in  figuur  \ref{fig:lijnVV}

\figuur[0.6]{lijnVV}{De lijn  $ l:  \ y = - \frac{1}{2}x + \frac{5}{2} $. Let op de steunvector (\red{rood}) en de richtingsvector (zwart)}

\subsubsection{Van vergelijking lijn naar vectorvoorstelling lijn:}
Als we, omgekeerd, van een vergelijking een vectorvoorstelling willen maken moeten we zorgen dat er een parameter $ \lambda $ in de vectorvoorstelling komt. Neem als voorbeeld de vergelijking $ y = 3x - 2 $. We weten dat we een $ \lambda $ moeten hebben (invoeren). Stel daarvoor dat $ y =  \lambda $ dan volgt uit de vergelijking  dat $ \lambda = 3x -2 $ dus $ x = \frac{1}{3}  \lambda  + \frac{2}{3} $. Dan hebben we  2 vergelijkingen, een voor y en een voor x:

\[\begin{cases}
x = \frac{2}{3} + \frac{1}{3} \lambda\\
y =  \lambda 
\end{cases}
\] 
Dat schrijven we dan met behulp van vectoren. 
$ l: \  \vectwee {x}{y} \ = \ \vectwee {\frac{2}{3}}{0} \ +\  \lambda  \vectwee {1/3}{1}  $.\\
Ga na dat dit precies dezelfde lijn is als:
$ l: \  \vectwee {x}{y} \ = \ \vectwee {1}{1} \ +\  \lambda  \vectwee {1}{3}  $.\\ (immers we kunnen voor $ \lambda $ 1 kiezen, daarmee een nieuw steupunt, (1,1), uitrekenen en vervolgens $ \lambda $ 3 keer zo groot kiezen)\\
Nog anders geschreven:
$  l: \  \vectweesnel{x} =  \ \vectwee {1}{1} \ +\  \lambda  \vectwee {1}{3}  $ \qquad
of: $ l: \ \vec{x} = \ \vectwee{1}{1} \ + \ \lambda \vectwee {1}{3} $

\subsection{Berekeningen vlak}
\subsubsection{Van vergelijking vlak naar vectorvoorstelling vlak:}
We hebben gezien dat $ ax + by + cz = d $ een vergelijking van een vlak in \RD is. Net zo als bij een lijn moeten we om een vectorvoorstelling van een vergelijking te maken parameters invoeren (bij een vlak hebben we 2 parameters nodig omdat een vlak  twee-dimensionaal is). Die parameters noemen we $ \lambda $  en $ \mu $ . Neem als voorbeeld het vlak $ -x + 2y -2z = 6 $.  We stellen  nu dat $ x=  \lambda $ en $ y = \mu $. Dan kunnen we $ z$ uitdrukken in $ \lambda $  en $ \mu $. Immers we vullen  $ x=  \lambda $ en $ y = \mu $ in in de vergelijking $ -x + 2y -2z = 6 $. Dan geldt dus: $ -\lambda + 2\mu -2z = 6 $, oftewel $ z = -3 + \mu -   \frac{1}{2} \lambda $. In feite hebben we nu 3 vergelijkingen namelijk:

\[\begin{cases}
x =  \lambda\\
y =  \mu\\
z = -3 + \mu -   \frac{1}{2} \lambda 
\end{cases}
\] 
anders geschreven:

\[\begin{cases}
x =  0 + \red{1}.\lambda +  \blu{0}.\mu\\
y =  0 +  \red{0}.\lambda  + \blu{1}. \mu\\
z = -3 \  \red{- \frac{1}{2}} .\lambda + \blu{1}.\mu 
\end{cases}
\] 
en dat kunnen we weer schrijven als:
$ V: \ \vecdrie{x}{y}{z} \ = \ \vecdrie{0}{0}{-3} \ + \ \lambda \  \vecdrie[red]{1}{0}{-\nicefrac{1} {2}}   \ + \ \mu \ \vecdrie[blue]{0}{1}{1} $ wat weer hetzelfde vlak is als waar we mee begonnen $ V: \ -x + 2y -2z = 6 $.
De vectorvoorstelling van een vlak kent dus \textit{twee}  richtingsvectoren. 
Zie  figuur  \ref{fig:vlak}.

\subsubsection{Van vectorvoorstelling vlak naar vergelijking vlak:}
Omgekeerd nemen we de vectorvoorstelling van een vlak $ V $,\\ 
bijvoorbeeld:
$ V: \  \vec{x} =\  \vecdriesnel{x} \ = \  \ \vecdrie{1}{3}{6} \ + \ \lambda \ \vecdrie{0}{2}{1} \ + \ \mu \ \vecdrie{1}{1}{2} $.\\
Om nu een vergelijking te krijgen moeten we $ \lambda $  en $ \mu $ wegwerken (elimineren). Dat kan als we zien dat de vectorvoorstelling van $ V $ eigenlijk 3 vergelijkingen bevat:

\[\begin{cases}
x_1 =  1 + 0\lambda + 1\mu\\
x_2 =  3 + 2\lambda  + 1\mu\\
x_3 = 6 + 1 \lambda + 2\mu  
\end{cases}
\] 
Uit de $ 1^{e}$ vergelijking halen we $ \mu = x_1 -1 $ en dat vullen we in in de $ 2^{e}$ en $ 3^{e}$ vergelijking:

\[\begin{cases}
x_2 =  3 + 2\lambda  + ( x_1 -1)\\
x_3 = 6 + 1 \lambda + 2( x_1 -1)
\end{cases}
\] 
Uit de $ 2^{e}$ vergelijking halen we dat $ \lambda = x_3 - 4 - 2x_1 $ wat we in de $ 1^{e}$ vergelijking invullen:
$ x_2 = 2 + 2(x_3 -4 -2x_1) + x_1 - 1$, anders geschreven: $ 3x_1 +x_2 - 2x_3 = -7
$. En dat is precies de algemene vorm van de vergelijking van een vlak.


\section{Normaalvectoren}
\subsection{Normaal van lijn}

Om goed te kunnen rekenen met lijnen hebben we een normaalvector nodig.
\mydef
{normaalvector lijn}
{De normaal of normaalvector van een lijn is: \\
	de vector $ \vec{n} $  \ die loodrecht staat op de richtingsvector van de lijn.}

\mybv[normaal]{Voorbeeld van een berekening van de normaal van een  lijn} 
Stel we hebben de lijn $ l: \  \vectwee {x}{y} \ = \ \vectwee {2}{0} \ +\  \lambda  \vectwee {3}{2}  $. (Zie figuur  \ref{fig:lijnnormaal})
Dan is de richtingsvector van $ l:  \overrightarrow{rv_{l}} =  \vectwee {3}{2} $. Misschien zie je meteen dat $ \vec{n} =  \vectwee {-2}{3} $ er loodrecht op staat? Ga na dat  $ (\overrightarrow{rv_{l} }, \vec{n}) = 3.-2 + 2.3  = 0 $. Als je dat niet meteen 'ziet', dan kun je het volgende doen: elke vector  $ \ne \vec{0} $ kun je schrijven als  $ \vec{n} = \vectwee {1}{c} $ waar c het getal is dat we zoeken. Er moet gelden dat  $ (\overrightarrow{rv_{l} }, \vec{n}) = 0 $, immers als twee vectoren loodrecht op elkaar staan moet het inproduct = 0 zijn. Dus $ 3.1 + 2.c = 0 $. Daaruit volgt $ c =  -\frac{3}{2}. $ en dus is $ \vectwee {1}{-3/2} $ de vector die we zoeken. Omdat voor het loodrecht zijn het niet uitmaakt hoe lang de vector is mogen we met -2 vermenigvuldigen om de breuk weg te halen: $ \vec{n} = -2.\vectwee {1}{\nicefrac{-3}{2}} \ = \  \vectwee {-2}{3} $. 

\figuur[0.4]{lijnnormaal}{De normaal $  \vec{n}  $  staat loodrecht op de richtingsvector van $ l $ }	

\subsection{Normaal van vlak}

Om goed te kunnen rekenen met vlakken hebben we een normaalvector nodig.
\label{vlaknormaal}
\mydef[]
{normaalvector vlak}
{De normaal of normaalvector van een vlak $\vec{n}$ of  $\overrightarrow{n_{V}}$ is:
	\\ de vector $ \vec{n} $  \ die loodrecht staat op \textit{beide} richtingsvectoren van het vlak. \\ 
	Gelukkig hebben we daarbij ook nog de volgende handige eigenschap:}

\myeig[normaal vlak]
{als $ V:  ax_1 + bx_2 +cx_3 = d $ \qquad dan is :  $ \vec{n} = \vecdrie{a}{b}{c} $ \ de normaalvector van 
	$ V $ .}

\mybv[normaal vlak]
{als $ V: 7x_1 + 3x_2 - x_3 = -20 $ \qquad dan is 
	$ \vec{n} = \vecdrie{7}{3}{-1} $ de normaalvector van $V$.\\}
Maar wat als we niet de vergelijking maar de vectorvoorstelling van een vlak hebben? Dan kunnen we gebruik maken van de eigenschap dat  het uitproduct van 2 vectoren loodrecht staat staat op beide richtingsvectoren. 

\mybv[normaal vlak]
{als $ W: \vec{x}  = \vecdrie{0}{1}{-1}    + \  \lambda \  \vecdrie{2}{-3}{1} + \ \mu \ \vecdrie{0}{1}{2} $ 
	\qquad dan zijn de richtingsvectoren $ \vecdrie{2}{-3}{1} \ $ en $  \vecdrie{0}{1}{2} $ \\
	en hun  uitproduct is : 
	$ \vecdrie{2}{-3}{1} \ \times \  \vecdrie{0}{1}{2}  \ = \ \vecdrie{-3.2-1.1}{-(2.2-1.0)}{-3.0-2.1} = \ \vecdrie{-7}{-3}{-2} $\\
	en dus is $ \vec{n} = \vecdrie{-7}{-3}{-2}  $ de normaal vector van $ W $.  Zie figuur  \ref{fig:vlaknormaal}}.

\figuur[0.8]{vlaknormaal}{De normaal van vlak \textit{W},  \ $  \vec{n}   \ = \ \vec{v} \times \vec{w} $ ,  staat loodrecht op de richtingsvectoren $  \vec{v}  $  en $  \vec{w} $}

De vergelijking van $ W $ is dan  $ -7x-4y-2z = c $ De constante c weten nog niet maar kunnen we uitrekenen doordat we weten dat het punt (0,1,-1) op W ligt, want dat is de vaste vector. (dat kun je ook zien als je en $ \lambda $  en $ \mu $  beide 0 stelt: dan is het punt (0 +0.2 +0.0, 1 -3.0 +1.0, -1 + 1.0 + 2.0)= (0,1,-1)  ) .
Vul de x, y en z-waarde van het vaste punt  in in de vergelijking $ -7x-4y-2z = c $ en er volgt dat c = -7.0-4.1-2.-1 = -2. Dus $ W: -7x-4y-2z = -2 $ , of, wat op hetzelfde neerkomt:  $ W: 7x+4y+2z = 2 $

\section{Afstand van punt tot vlak}
We hebben nu genoeg hulpmiddelen om in \RD de afstand van een punt tot een vlak uit te kunnen rekenen. Dat doen we   door een lijn \textit{l} loodrecht op \textit{V} te tekenen,\textit{ l }snijdt \textit{V} in\textit{ S} en dan is de afstand tussen \textit{P} en \textit{V} gelijk aan de afstand tussen P\textit{} en \textit{S}. Zie figuur   \ref{fig:afstand1}. en \ref{fig:afstand-2}.

Dat gaat in een aantal stappen:
\begin{enumerate}[label=(\alph*)]
	\item de vergelijking van een vlak bepalen
	\item de normaal van een vlak berekenen
	\item de lijn berekenen die  loodrecht op het vlak staat  en door $ P $ gaat
	\item x, y en z van de lijn $  l $  uitdrukken in \ $  \lambda $
	\item het snijpunt $\it{S}$  van de lijn door $\it{P}$  en het vlak bepalen
	\item  de afstand tussen $\it{P}$  en het snijpunt $\it{S}$  uitrekenen
\end{enumerate}	

\figuur[1]{afstand1}{Gevraagd: de afstand tussen $ P $ en $ V $, ( $ P $ is een punt dat boven het vlak 'zweeft')}
\mybv[afstand]
{ We nemen als voorbeeld voor de berekening \\het punt $ P: (3,6,7)  $ en het vlak 
	$ V: \ \vecdrie{x}{y}{z} \ = \ \vecdrie{5}{0}{3} \ + \ \lambda \ \vecdrie{-1}{-1}{1} \  + \mu \ \vecdrie{3}{5}{-6} $ \ \ \ .
}

\begin{enumerate}[label=(\alph*)]
\item \subsubsection{de vergelijking van een vlak beplen}
Omdat de vectorvoorstelling van het vlak gegeven is, moeten we daarvan eerst een vergelijking maken.
We vinden met de methode van de paragraaf "normaal van vlak" \ 
(blz. \pageref{vlaknormaal}) dat : $ V: -x + 3y +2z = 1 $.  Als de vergelijking van het vlak al gegeven is kun je deze stap overslaan natuurlijk. 
\item \subsubsection{de normaal van een vlak berekenen}
Als we de vergelijking van het vlak hebben is het berekenen van de normaal kinderspel. Immers de normaal wordt gegeven door de getallen die in de vergelijking van het vlak staan:
$ \vec{n} = \vecdrie{-1}{3}{2}  $ de normaal vector van $ V $.\\
NB: Let op: De normaal kun je alleen uit de vergelijking van het vlak aflezen als "\textit{de x, y en z aan de linkerkant van het = - teken staan}".  Dus als je als vergeijking $ 3x + z = 2 -y $ hebt, dan is de normaal NIET $  \vecdrie{3}{-1}{1}  $ maar  $ \vecdrie{3}{1}{1}  $, omdat uit $ 3x + z = 2 -y $ volgt dat $ 3x + y + z = 2 $ (x, y en z aan de linkerkant).
\item \subsubsection{de lijn berekenen die  loodrecht op het vlak staat  en door $ P $ gaat}
De lijn door het punt P die loodrecht op vlak V staat heeft als richtingsvector . . .  de normaal van het vlak. Immers de normaal staat loodrecht op dat vlak. Dus $ \overrightarrow{rv_{l}}  = \vec{n} $. En de lijn moet door $ P: (3,6,7)  $  gaan. Dat  betekent dat we $ \vecdrie{3}{6}{7} $ als steunvector voor de lijn kunnen gebruiken. Dus de vectorvoorstelling van de lijn is: $ l: \ \vecdrie{x}{y}{z} \ = \ \vecdrie{3}{6}{7} \ + \ \lambda \ \vecdrie{-1}{3}{2} \ $
\item \subsubsection{x, y en z van de lijn $  l $  uitdrukken in \ $  \lambda $}
Net zoals bij een vlak in \RD bestaat de vectorvorstelling van de lijn $ l  $ eigenlijk uit 3 vergelijkingen:
\[\begin{cases}
x =  3 -\lambda \\
y =  6 + 3\lambda\\
z = 7 + 2\lambda
\end{cases}
\] 
\item \subsubsection{het snijpunt van  het  vlak en de lijn door P bepalen}
Voor het snijpunt van\textit{ l} en \textit{V} gedlt dat het snijpunt zowel op $  l $ ligt als op $ V $. Anders gezegd: De coordinaten  van het snijpunt moeten voldoen aan bovenstaande 3 vergelijingen van\textit{ l}, maar ook aan de vergelijking voor \textit{V}. Met andere woorden: we kunnen de x, y en z van hierboven invullen in de vergelijking van het vlak:\\
$ -(3 -\lambda) + 3.(6 + 3\lambda) + 2.(7 + 2\lambda) = 1$ \\
of $ 14\lambda + 29 = 1 $ \\of $  \lambda = -2 $

\item \subsubsection{de afstand uitrekenen}
We weten nu dat als $  \lambda = -2 $ we het snijpunt van $ l  $ en $ V $ hebben. Wat betekent dat? Dat betekent dat als we beginnen in punt P en -2 keer de richtingsvector van $  l $ er bij op tellen we op het vlak V terecht komen.
Anders gezegd de afstand tussen P en V is 2 keer de lengte van die richtingsvector. Oftewel: 
afstand $ = 2. |\overrightarrow{rv_{l}})| = 2. \sqrt{(-1)^{2} + (3)^{2} + (2)^{2} } = 2\sqrt{14} $ \. Zie  figuur  \ref{fig:afstand-2}
\end{enumerate}	

\figuur[1]{afstand-2}{We moeten vanuit $ P $ beginnend -2 keer de vector $\vec{n} =  \vec{rv_{l}} $   afpassen om op V te komen. Anders gezegd:  de normaal vector van $ V $ past  precies 2 keer tussen $ V $ en $ P $, en dus is de afstand $ 2.|\vec{n}| =  2.|\vec{rv_{l}}| = 2\sqrt{14} $.}


\subsubsection{Opgaven}
\begin{enumerate}
	\item  Bereken \ $ \vec{a} + \vec{b} $, \quad $ \vec{a} - \vec{b} $, \ en \ $  (\vec{a} , \vec{b}) $  als $ \vec{a} = \vecvier{3}{-1}{2}{0}  $ \  en \  $  \vec{b} =  \vecvier{-2}{2}{4}{7} $
	
	\item  Bereken de hoek $\alpha$ die de  lijnen \ $  l $ en $ m $ met elkaar maken in één decimaal nauwkeurig:
	$ l:  y = 2x +7  $ en $ m: \  \vectwee{x}{y} = \vectwee{-3}{7} + \lambda \vectwee{2}{-3}  $
	
	\item Wat is de lengte van de vector $  \vec{v} \ = (2, -3, 0, 2, -5) $ ?
	% \ \vecvijf{2}{-3}{0}{2}{-5} $ ?
	
	\item Schaal de vector $  \vec{v} $ naar de eenheidsvector $\hat{v}$
	als $  \vec{v} = (-1, 5, 7, 0, 5) $.
	%\vecvijf{-1}{5}{7}{0}{5} $ 
	
	\item Bereken de afstand tussen  $  P = (7,0,-2)  $ en 
	$ V: \ \vecdriesnel{x} \
	=  \vecdrie {8} {-5} {3}  \
	+  \ \lambda \ \vecdrie{3}{-2}{1} 
	+ \mu \  \vecdrie{-5}{3}{-2} $ 
	
	\item   Bereken de afstand tussen  $  Q = (7,1,2)  $ en 
	$ W:2x - 4y +2z \ = -2 $ .
	
\end{enumerate}

\subsubsection{Extra opgaven}
\begin{enumerate}
	\item Bereken 
	\ $ \vec{d} + \vec{e} $, \ $ \vec{d} - \vec{e} $, en \ $  (\vec{d} , \vec{e}) $ \ als 
	\quad $ \vec{d} = \vecvier{3}{-2}{0}{5}  $ \  en \  $  \vec{e} =  \vecvier{1}{7}{-2}{2} $ 	
	
	\item  Bereken de hoek $\alpha$ die de  lijnen \ $  m $ en $ l $ met elkaar maken in één decimaal nauwkeurig:
	$ m:  y = 3x + 1  $ en $ l: \  \vectwee{x}{y} = \  \lambda \vectwee{4}{-3}  $
	
	\item Wat is de lengte van de vector $  \vec{w} \ = \ \vecvijf{4}{0}{7}{-4}{1} $ ?
	
	\item Bereken de afstand tussen  $  P = (-3,9,-3)  $ en 
	$ V: \ \vecdriesnel{x} \ =  \ \lambda \ \vecdrie{4}{2}{4} + \mu \  \vecdrie{7}{-1}{-3} $	
	
	\item   Bereken de afstand tussen  $  Q = (-1,-4,-1)  $ en 
	$ W: -2x + 9y + 6z \ = \nicefrac{1}{3} $ .
\end{enumerate}

